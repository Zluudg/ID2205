%%%%%%%%%%%%%%%%%%%%%%%%%%%%%%%%%%%%%%%%%%%%%%%%%%%%%%%%%%%%%%%%%%%%%%%
%% template for II2202 report
%% original 2015.11.24
%% revised  2018.08.26
%%%%%%%%%%%%%%%%%%%%%%%%%%%%%%%%%%%%%%%%%%%%%%%%%%%%%%%%%%%%%%%%%%%%%%%
%

\title{ID2205 Project Plan for 'A Development Environment for DMDL'}
\author{
        \textsc{Leon Fernandez}
            \qquad
        \textsc{Peng Wang}
        \mbox{}\\
        \normalsize
            \texttt{leonfe}
        \textbar{}
            \texttt{pengwang}
        \normalsize
            \texttt{@kth.se}
}
\date{\today}

\documentclass[12pt,twoside]{article}

\usepackage[paper=a4paper,dvips,top=1.5cm,left=1.5cm,right=1.5cm,
    foot=1cm,bottom=1.5cm]{geometry}

\usepackage{enumitem}
\usepackage{amsfonts}
%\usepackage[T1]{fontenc}
%%\usepackage{pslatex}
\renewcommand{\rmdefault}{ptm} 
\usepackage{mathptmx}
\usepackage[scaled=.90]{helvet}
\usepackage{courier}
%
\usepackage{bookmark}

\usepackage{fancyhdr}
\pagestyle{fancy}

%------ GANTT PACKAGE --------------------------------------------------------
\usepackage{pgfgantt}
%-----------------------------------------------------------------------------

%% \usepackage{pgfmath}	% --math engine
%%----------------------------------------------------------------------------
%% \usepackage[latin1]{inputenc}
\usepackage[utf8]{inputenc} % inputenc allows the user to input accented characters directly from the keyboard
\usepackage[swedish,english]{babel}
%% \usepackage{rotating}		 %% For text rotating
\usepackage{array}			 %% For table wrapping
\usepackage{graphicx}	                 %% Support for images
\usepackage{float}			 %% Suppor for more flexible floating box positioning
\usepackage{color}                       %% Support for colour 
\usepackage{mdwlist}
%% \usepackage{setspace}                 %% For fine-grained control over line spacing
%% \usepackage{listings}		 %% For source code listing
%% \usepackage{bytefield}                %% For packet drawings
\usepackage{tabularx}		         %% For simple table stretching
%%\usepackage{multirow}	                 %% Support for multirow colums in tables
\usepackage{dcolumn}	                 %% Support for decimal point alignment in tables
\usepackage{url}	                 %% Support for breaking URLs
\usepackage[perpage,para,symbol]{footmisc} %% use symbols to ``number'' footnotes and reset which symbol is used first on each page

%% \usepackage{pygmentize}           %% required to use minted -- see python-pygments - Pygments is a Syntax Highlighting Package written in Python
%% \usepackage{minted}		     %% For source code highlighting

%% \usepackage{hyperref}		
\usepackage[all]{hypcap}	 %% Prevents an issue related to hyperref and caption linking
%% setup hyperref to use the darkblue color on links
%% \hypersetup{colorlinks,breaklinks,
%%             linkcolor=darkblue,urlcolor=darkblue,
%%             anchorcolor=darkblue,citecolor=darkblue}

%% Some definitions of used colors
\definecolor{darkblue}{rgb}{0.0,0.0,0.3} %% define a color called darkblue
\definecolor{darkred}{rgb}{0.4,0.0,0.0}
\definecolor{red}{rgb}{0.7,0.0,0.0}
\definecolor{lightgrey}{rgb}{0.8,0.8,0.8} 
\definecolor{grey}{rgb}{0.6,0.6,0.6}
\definecolor{darkgrey}{rgb}{0.4,0.4,0.4}
%% Reduce hyphenation as much as possible
\hyphenpenalty=15000 
\tolerance=1000

%% useful redefinitions to use with tables
\newcommand{\rr}{\raggedright} %% raggedright command redefinition
\newcommand{\rl}{\raggedleft} %% raggedleft command redefinition
\newcommand{\tn}{\tabularnewline} %% tabularnewline command redefinition

%% definition of new command for bytefield package
\newcommand{\colorbitbox}[3]{%
	\rlap{\bitbox{#2}{\color{#1}\rule{\width}{\height}}}%
	\bitbox{#2}{#3}}

%% command to ease switching to red color text
\newcommand{\red}{\color{red}}
%%redefinition of paragraph command to insert a breakline after it
\makeatletter
\renewcommand\paragraph{\@startsection{paragraph}{4}{\z@}%
  {-3.25ex\@plus -1ex \@minus -.2ex}%
  {1.5ex \@plus .2ex}%
  {\normalfont\normalsize\bfseries}}
\makeatother

%%redefinition of subparagraph command to insert a breakline after it
\makeatletter
\renewcommand\subparagraph{\@startsection{subparagraph}{5}{\z@}%
  {-3.25ex\@plus -1ex \@minus -.2ex}%
  {1.5ex \@plus .2ex}%
  {\normalfont\normalsize\bfseries}}
\makeatother

\setcounter{tocdepth}{3}	%% 3 depth levels in TOC
\setcounter{secnumdepth}{5}
%%%%%%%%%%%%%%%%%%%%%%%%%%%%%%%%%%%%%%%%%%%%%%%%%%%%%%%%%%%%%%%%%%%%
%% End of preamble
%%%%%%%%%%%%%%%%%%%%%%%%%%%%%%%%%%%%%%%%%%%%%%%%%%%%%%%%%%%%%%%%%%%%
\renewcommand{\headrulewidth}{0pt}
\lhead{II2202, Fall 2018, Period 1-2}
%% or \lhead{II2202, Fall 2016, Period 1}
\chead{Draft project report}
\rhead{\date{\today}}
\makeatletter
\let\ps@plain\ps@fancy 
\makeatother
\setlength{\headheight}{15pt}
\begin{document}


\maketitle


\begin{abstract}
\label{sec:abstract}

This document is the project plan for the project course ID2205 Individual Advanced Studies in Software Systems at KTH Royal Institute of Technology in Stockholm.
The project will consist of developing a web-based Development Environment (DE) for the graphical programming language Decomposite MAC Description Language (DMDL).
\end{abstract}


\clearpage


\selectlanguage{english}


\tableofcontents


\section*{List of Acronyms and Abbreviations}
\label{sec:acronyms}

This document requires readers to be familiar with terms and concepts described in \cite{dmdl}. For clarity some basic terms are here summarized for ease of reference.
\begin{basedescript}{\desclabelstyle{\pushlabel}\desclabelwidth{10em}}
\item[MAC] Medium Access Control, a sublayer of the second layer of the OSI model\cite{osi}
\item[DMDL] Decomposite MAC Description Language, a graphical programming language for describing MAC protocols
\end{basedescript}


\clearpage


\section{Background}
\label{sec:background}

Increasing demands on wireless communication systems means increasing demands on wireless system development. Several tools and technologies are used
to to speed up the development cycle, most notably the Software Defined Radio (SDR) \cite{sdr}. The SDR is a core concept in many tools that support
digital radio development, such as GNU Radio and LabVIEW. However, most of these tools are either platform-dependent or
focused on the development of Physical (PHY) layer protocols. New development tools for MAC layer protocols are needed in order to fully utilize
the flexibility provided by SDRs.

DMDL \cite{dmdl} is a graphical programming language aimed at achieving a swift prototyping and
development workflow for MAC protocols. DMDL decomposes standard MAC layer operations such as carrier sensing and address lookup into functional blocks
that can be connected together and modeled acording to the Synchronous Data Flow (SDF) model of computation.

\section{Goals}
\label{sec:goals}

\subsection{Learning Outcome}
The expected learning outcome of the project is a deeper understanding of SDRs and MAC protocol simulation and prototyping in
preparation for a possible thesis collaboration in the future.

\subsection{Deliverables}
The project is expected to deliver a lightweight, web-based DE for DMDL. The DE need not
be able to interface with actual SDR hardware at the end of the project. It shall, however, be able to perform a syntax check
of a program to determine whether that program describes a nonsensical MAC protocol or not.


\section{Organization}
\label{sec:organization}

The project serves as the content in the course ID2205 Individual Advanced Studies in Software Systems at KTH Royal Institute of Technology in Stockholm. 
Table \ref{tab:people} lists all people involved in the project.
\begin{table}[!h]
\renewcommand{\arraystretch}{1.3}
\caption{The people involved in the project}
\label{tab:people}
\centering
\begin{tabular}{|c|c|c|}
\hline
\textbf{Name} & \textbf{Contact} & \textbf{Role}\\
\hline
Leon Fernandez & leonfe@kth.se & Student, Main Developer\\
\hline
Peng Wang & pengwang@kth.se & Project Owner\\
\hline
Marina Petrova & petrovam@kth.se & Supervisor\\
\hline
Christian Schulte & cschulte@kth.se & Course Examiner\\
\hline
\end{tabular}
\end{table}

\section{Project Overview}
\label{sec:overview}

The project will constist of three main phases for development, listed in the subsections below. Time
plan details can be found in Section \ref{sec:commentary} and Appendix \ref{sec:timeplan}.

\subsection{Phase 1: Block Design}
The focus of the first phase will be on the development of the functional blocks that make up the language
and the setup of the DEs GUI. The project owner offers a set of template icons that can be used when
designing the blocks. The developed blocks should have a panel which can be opened via double click or right click
so that the properties of the block can be set. A property might for instance be the duration if the block is a timer block.
The blocks and their proprties can be found in the DMDL documentation [\textit{INSERT REFERENCE HERE}].
\subsubsection{Completion Criteria}
\begin{itemize}[label={\checkmark}]
\item A set of blocks that immplement the MAC operations defined by the DMDL language
\item A canvas where the blocks can be placed
\item A panel from which the user can select a block to place
\end{itemize}

\subsection{Phase 2: Connection Design}
The second phase will consist of developing the mechanisms needed to connect
the blocks together. This means assigning ports to the blocks which can then be connected using the mouse.
Ports should be divided into two categories, namely input ports and output ports. Port lists for all blocks
in the DMDL language can be found in the DMDL documentation [\textit{INSERT REFERENCE HERE}].
Depending on how the properties of a certain block are set, its port might change. The DE should
be able to handle this by being able to hide unused ports. Illegal port connections, such as connecting
and output port to an output port, shall be marked with an error.
\subsubsection{Completion Criteria}
\begin{itemize}[label={\checkmark}]
\item The user can use the mouse to connect the ports of the blocks into a SDF graph
\item Illegal port connections are highlighted as an error
\end{itemize}

\subsection{Phase 3: Syntax Check}
The aim of the third phase is to implement a mechanism for performing a simple syntax check of the user's program.
Unlike traditional languages, DMDL was specifically designed to describe and handle MAC
layer protocols. Therefore, the DE should be able to perform a check to verify that the user's program does indeed describe a sensible MAC
protocol. Non-sensible MAC protocol behavior might for instance be having a "Send"-block but no "Generate Frame"-block. The DE
should provide the user with helpful feedback for such errors. 
\subsubsection{Completion Criteria}
\begin{itemize}[label={\checkmark}]
\item A button for performing a syntax check 
\item A panel for displaying error feedback
\end{itemize}

\subsection{Additional Completion Criteria}
In addition to the criteria mentioned above the project is also expected to deliver:
\begin{itemize}[label={\checkmark}]
\item A well-structured project report that adheres to the course guidelines
\item A well-documented repository containing all self-written code
\item A 20 minute oral presentation and demonstration of the project and the DE
\end{itemize}

\section{Timeplan Commentary}
\label{sec:commentary}
\textit{To be written}


%\section{Risks}
%\label{sec:risks}


\section{Project Documentation}
\label{documentation}

The project will be documented in the form of a git repository. All project documentation will be written
in LaTeX and the corresponding source code will be version handled using git. The source code for the DE will
be version handled in the same way and in the same repository. Documentation for the DE source code
will be generated using Doxygen. The project owner and the student will meet on a weekly basis and the
contents of the meeting will be breifly summarized in a short report that will be available on the git repository as well. 


\clearpage

\bibliography{ID2205_Leon_Fernandez_Project_Plan}
\bibliographystyle{IEEEtran}

\clearpage

\appendix
\section{Timeplan}
\label{sec:timeplan}

\begin{ganttchart}{1}{12}
\gantttitle{2011}{12} \\
\gantttitlelist{1,...,12}{1} \\
\ganttgroup{Group 1}{1}{7} \\
\ganttbar{Task 1}{1}{2} \\
\ganttlinkedbar{Task 2}{3}{7} \ganttnewline
\ganttmilestone{Milestone}{7} \ganttnewline
\ganttbar{Final Task}{8}{12}
\ganttlink{elem2}{elem3}
\ganttlink{elem3}{elem4}
\end{ganttchart}


\end{document}
