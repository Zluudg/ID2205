\documentclass{article}
\usepackage[utf8]{inputenc} %codification of the document
 
\usepackage{hyperref}
 
%Here begins the body of the document
\begin{document}

\section{November 6th}
I have decided to base the dmdl-editor on the \texttt{mxgraph} library, which seems suitable for the situation.
It is available on https://github.com/jgraph/mxgraph and licensed under Apache V2.0, which I hope is fine (I asked
Peng about it in an email today but he has not answered yet). Most of my day was spent studying the basics of 
JavaScript and creating a basic skeleton structure for the application.

\section{November 7th}
I looked into the Canvas API featured in HTML 5 as a simpler alternative to mxgraph. After spending the day with mxgraph
I realized that the API is a bit too complicated for me and that I am probably better of using what is readily available in
plain old JavaScript, HTML and CSS. The first version of the editor that I intent to implement will feature a dropdown menu
with 3-4 blocks, an area for placing the blocks and a configuration menu for the selected instance of a block.

\section{November 10th}
I started looking into a guide \href{https://simonsarris.com/making-html5-canvas-useful/}{here}.
The approach is rather boilerplate, which I think is good for getting the basics down.

\section{November 11th}
Simple dummy blocks can now be added with the editor by doubleclicking. For now
all blocks are of type ''generic'' and has a fixed number of ports. I intend to make the port
count configurable in the rightmost menu by bringing up a panel for the selected block where
the user can add and remove ports, among other things. However, the next step will be
to extend the generic Block class with some real classes that the user can select
between in the leftmost menu. That means the three main HTML elements will have to interact.

Most of the work so far follows the guide mentioned in the previous entry.
\end{document}


